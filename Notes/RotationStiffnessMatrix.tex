\documentclass[12pt,a4paper]{article}

\usepackage[english]{babel}
\usepackage[utf8]{inputenc}
\usepackage{amsmath}
\usepackage{mathrsfs}
\usepackage{epsfig}


\begin{document}

\title{Rotation of Stiffness Matrix for Anisotropic Materials}

\section{Coordinate system and principal axis}

\begin{figure}
	\epsfig{file=EulerAngles.eps, scale=0.55}
	\caption{Description of Euler angles}
	\label{fig:EulerAngles}     
\end{figure}

\subsection{Rotation matrix}

\begin{equation}
\label{eq:RotMat}
R = 
\begin{bmatrix}
  \cos{\phi}\cos{\psi}-\sin{\phi}\cos{\theta}\sin{\psi} & \sin{\phi}\cos{\psi}+\cos{\phi}\cos{\theta}\sin{\psi} & 
  \sin{\theta}\sin{\psi}\\ 
  -\cos{\phi}\sin{\psi} - \sin{\phi}\cos{\theta}\cos{\psi}& 
  -\sin{\phi}\sin{\psi} + \cos{\phi}\cos{\theta}\cos{\psi}& 
  \sin{\theta}\cos{\psi}  \\
  \sin{\phi}\sin{\theta} & 
  -\cos{\phi}\sin{\theta} &
  \cos{\theta} \\ 
\end{bmatrix} 
\end{equation}

\begin{equation}
\label{eq:BondMat}
M = 
\begin{bmatrix}
R_{11}^2 & R_{12}^2 & R_{13}^2 & 2 R_{12} R_{13} & 2 R_{13} R_{11} & 2 R_{11} R_{12} \\ 
R_{21}^2 & R_{22}^2 & R_{23}^2 & 2 R_{22} R_{23} & 2 R_{23} R_{21} & 2 R_{21} R_{22} \\
R_{31}^2 & R_{32}^2 & R_{33}^2 & 2 R_{32} R_{33} & 2 R_{33} R_{31} & 2 R_{31} R_{32} \\
R_{21} R_{31} & R_{22} R_{32} & R_{23} R_{33} & R_{22} R_{33} + R_{23} R_{32} & R_{21} R_{33} + R_{23} R_{31} & R_{22}R_{31} + R_{21} R_{32} \\
R_{31} R_{11} & R_{32} R_{12} & R_{33} R_{13} & R_{12} R_{33} + R_{13} R_{32} & R_{13} R_{31} + R_{11} R_{33} & R_{11}R_{32} + R_{12} R_{31} \\  
R_{11} R_{21} & R_{12} R_{22} & R_{13} R_{23} & R_{12} R_{23} + R_{13} R_{22} & R_{13} R_{21} + R_{11} R_{23} & R_{11}R_{22} + R_{12} R_{21} \\    
\end{bmatrix} 
\end{equation}


\section{Olivier's implementation}
We will omit all the piezoelectric and magnetic properties and reverse engineer the formulas. Few helpful things first:

\begin{equation}
\label{eq:GammaMatrix}
\Gamma_{ij}= 
\begin{bmatrix}
C_{i11l} & C_{i12l} & C_{i13l} \\
C_{i21l} & C_{i22l} & C_{i23l} \\
C_{i31l} & C_{i32l} & C_{i33l} \\
\end{bmatrix} 
\end{equation}
This can be converted using the Voigt's notation to:
\begin{equation}
\label{eq:GammaMatrixVoigt}
\Gamma_{11}=
\begin{bmatrix}
C_{1111} & C_{1121} & C_{1131} \\
C_{1211} & C_{1221} & C_{1231} \\
C_{1311} & C_{1321} & C_{1331} \\
\end{bmatrix}=  
\begin{bmatrix}
C_{11} & C_{16} & C_{15} \\
C_{61} & C_{66} & C_{65} \\
C_{51} & C_{56} & C_{55} \\
\end{bmatrix} 
\end{equation} 
Now we can start constructing the Stroh matrix step by step
\begin{equation}
\label{eq:A1}
A_1 = A_1'+A_2''A_{BC},  
\end{equation}
where
\begin{align}
\label{eq:A1prime}
	A_1' = &(\Gamma_{11}-\Gamma_{13}\Gamma_{33}^{-1}\Gamma_{31})\cos^2{\psi'} \nonumber\\ &+(\Gamma_{12}+\Gamma_{21}-\Gamma_{13}\Gamma_{33}^{-1}\Gamma_{32}-\Gamma_{23}\Gamma_{33}^{-1}\Gamma_{31})\sin{\psi'}\cos{\psi'} \nonumber\\
	&+(\Gamma_{22}-\Gamma_{23}\Gamma_{33}^{-1}\Gamma_{32})\sin^2{\psi'}
\end{align}
and 
\begin{equation}
\label{eq:A2prime}
A_2' =\Gamma_{33}^{-1}(\Gamma_{31} \cos{\psi'} + \Gamma_{32} \sin{\psi'})  
\end{equation}
and the last part
\begin{equation}
\label{eq:A2dblprime}
A_2'' = (\Gamma_{13} \cos{\psi'} + \Gamma_{23} \sin{\psi'})\Gamma_{33}^{-1}  
\end{equation}
Now we have to start tuning the pieces in the formula. Let's start from the last one.
\begin{equation}
\label{eq:ABC}
A_{BC} = \Gamma_{33}A_2' = \Gamma_{33}\Gamma_{33}^{-1}(\Gamma_{31} \cos{\psi'} + \Gamma_{32} \sin{\psi'}) = \Gamma_{31} \cos{\psi'} + \Gamma_{32} \sin{\psi'}
\end{equation}
Now we can $A_{BC}$ and $A_2''$ to pre-calculate the second part of (\ref{eq:A1}) as follows:
\begin{align*}
	A_2'' A_{BC} = & (\Gamma_{13} \cos{\psi'} + \Gamma_{23} \sin{\psi'})\Gamma_{33}^{-1} (\Gamma_{31} \cos{\psi'} + \Gamma_{32} \sin{\psi'}) \nonumber\\
				 = & \Gamma_{13}\Gamma_{33}^{-1}\Gamma_{31} \cos^2{\psi'} + \Gamma_{13}\Gamma_{33}^{-1}\Gamma_{32} \cos{\psi'}\sin{\psi'} \nonumber \\
				 + & \Gamma_{23}\Gamma_{33}^{-1}\Gamma_{31}\cos{\psi'}\sin{\psi'}+ \Gamma_{23}\Gamma_{33}^{-1}\Gamma_{32} \sin^2{\psi'}
\end{align*}
Now, we can finally evaluate equation (\ref{eq:A1}) by plugging the previous result as the second term
\begin{align*}
	A_1= A_1'+A_2''A_{BC} = &(\Gamma_{11}-\Gamma_{13}\Gamma_{33}^{-1}\Gamma_{31})\cos^2{\psi'} \nonumber\\
				 + &(\Gamma_{12}+\Gamma_{21}-\Gamma_{13}\Gamma_{33}^{-1}\Gamma_{32} - \Gamma_{23}\Gamma_{33}^{-1}\Gamma_{31})\sin{\psi'}\cos{\psi'} \nonumber\\
				 + & (\Gamma_{22}-\Gamma_{23}\Gamma_{33}^{-1}\Gamma_{32})\sin^2{\psi'}\\
				 + &  \Gamma_{13}\Gamma_{33}^{-1}\Gamma_{31} \cos^2{\psi'} + \Gamma_{13}\Gamma_{33}^{-1}\Gamma_{32} \cos{\psi'}\sin{\psi'} \nonumber \\
				 + & \Gamma_{23}\Gamma_{33}^{-1}\Gamma_{31}\cos{\psi'}\sin{\psi'}+ \Gamma_{23}\Gamma_{33}^{-1}\Gamma_{32} \sin^2{\psi'} \nonumber \\
				 = & \Gamma_{11}\cos^2{\psi'} + (\Gamma_{12}+\Gamma_{21})\cos{\psi'}\sin{\psi'} + \Gamma_{22}\sin^2{\psi'}
\end{align*}
Another elements we have to calculate are $B$, $A_2$ and $C$. Let's start with $B$ which is defined as:
\begin{align}
	B = &\Gamma_{33}A_2' + A_2''\Gamma_{33} = \Gamma_{33} \Gamma_{33}^{-1}(\Gamma_{31} \cos{\psi'} + \Gamma_{32} \sin{\psi'}) \nonumber \\
		&+ (\Gamma_{13} \cos{\psi'} + \Gamma_{23} \sin{\psi'})\Gamma_{33}^{-1}\Gamma_{33} \nonumber \\
		& = (\Gamma_{31} + \Gamma_{13}) \cos{\psi'} + (\Gamma_{32} + \Gamma_{23})\sin{\psi'}
\end{align}
Now we can proceed to $A_2$
\begin{equation}
	A_2 = -\rho_0 \omega^2 I_p,
\end{equation}
where $\rho_0$ is layer density and $I_p$ identity matrix. At last, $C$ is defined simply as
\begin{equation}
	C = -\Gamma_{33}
\end{equation}
Recapitulation of the most important results. These will be the variables we'll be using further on:
\begin{eqnarray}
	A_1 &=& \Gamma_{11}\cos^2{\psi'} + (\Gamma_{12}+\Gamma_{21})\cos{\psi'}\sin{\psi'} + \Gamma_{22}\sin^2{\psi'} \nonumber \\
	B 	&=& (\Gamma_{31} + \Gamma_{13}) \cos{\psi'} + (\Gamma_{32} + \Gamma_{23})\sin{\psi'} \nonumber \\
	A_2 &=& -\rho_0 \omega^2 I_p \nonumber \\
	C	&=& -\Gamma_{33} \nonumber \\
	A_{BC} &=&  \Gamma_{31} \cos{\psi'} + \Gamma_{32} \sin{\psi'} \nonumber
\end{eqnarray}
In the implementation the $A_2$ is not multiplied by the $\omega^2$ in the preparatory stage.



\end{document}

